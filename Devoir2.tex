\documentclass[16pt]{report}
\usepackage[french]{babel}
% Permet d'ajuster la taille des marges et de la distance pour les footer
\usepackage[tmargin=2cm,rmargin=0.4in,lmargin=0.4in,bmargin=2cm,footskip=.5in]{geometry}

% Permet d'optimiser l'affichage de différents symboles et formules mathématiques
\usepackage{amsmath,amsfonts,amsthm,amssymb,mathtools}

\usepackage{svg}

% Modifie l'apparence des nombre en mathmode et textmode
%\usepackage[varbb]{newpxmath}

% Modifier l'apparence des fractions
\usepackage{xfrac}

            %%%%%%%%%%%%%%%%%  Sect.        14 Oct 2024     %%%%%%%%%%%%%%%%%%%%%%%%%%%%%%%%%%%%%%%%%%%%%%%%%%%%%%%%%%%
\usepackage{graphicx}
\usepackage{caption}
\usepackage{subcaption}
\usepackage{arydshln}
            %%%%%%%%%%%%%%%%%  Sect.        14 Oct 2024     %%%%%%%%%%%%%%%%%%%%%%%%%%%%%%%%%%%%%%%%%%%%%%%%%%%%%%%%%%%
\usepackage{balance}
\usepackage{dirtree}
\usepackage{titlesec}
\usepackage{blkarray}





% Permet de rayer (barrer) l'argument avec la touche
% \cancel{} \bcancel{} ou \xcancel{}
\usepackage[makeroom]{cancel}

% Extension du package amsmath; corrige certains bugs et déficiences de son prédecesseur
\usepackage{mathtools}

% This package provides most of the flexibility you may want to customize the three basic list
% environments (enumerate, itemize and description)
\usepackage{bookmark} 

% Réorganiser les théorèmes et Lemmes. Usage complexe. 
% Référence : https://ctan.math.illinois.edu/macros/latex/contrib/theoremref/theoremref-doc.pdf
\hypersetup{hidelinks}
\usepackage{hyperref,theoremref} 

% Fournit un environnement pour créer des boîtes colorées
\usepackage[most,many,breakable]{tcolorbox}


%\newcommand\mycommfont[1]{\footnotesize\ttfamily\textcolor{blue}{#1}}\SetCommentSty{mycommfont}

%\newcommand{\incfig}[1]{%\def\svgwidth{\columnwidth}\import{./figures/}{#1.pdf_tex}}
\newcommand{\arc}[1]{\wideparen{#1}}

%Pour colorer les lignes séparatrices de tableaux
\usepackage{colortbl}
\usepackage{tikzsymbols}

\usepackage{framed}
\usepackage{titletoc}
\usepackage{etoolbox}
\usepackage{lmodern}
\usepackage{tabularx}
\usepackage{enumitem}
\usepackage{amsthm}
            %%%%%%%%%%%%%%%%%  Sect.        14 Oct 2024     %%%%%%%%%%%%%%%%%%%%%%%%%%%%%%%%%%%%%%%%%%%%%%%%%%%%%%%%%%%

\usepackage{lipsum}
\usepackage{titling}
\renewcommand\maketitlehooka{\null\mbox{}\vfill}
\renewcommand\maketitlehookd{\vfill\null}

\newcommand{\varitem}[3][black]{%
  \item[%
   \colorbox{#2}{\textcolor{#1}{\makebox(5.5,7){#3}}}%
  ]
}
\usepackage{afterpage}
\newcommand\myemptypage{
    \null
    \thispagestyle{empty}
    \addtocounter{page}{-1}
    \newpage
    }




% from https://tex.stackexchange.com/a/167024/121799
\newcommand{\ClaudioList}{red,DarkOrange1,Goldenrod1,Green3,blue!50!cyan,DarkOrchid2}
\newcommand{\SebastianoItem}[1]{\foreach \X[count=\Y] in \ClaudioList
{\ifnum\Y=#1\relax
\xdef\SebastianoColor{\X}
\fi
}
\tikz[baseline=(SebastianoItem.base),remember
picture]{%
\node[fill=\SebastianoColor,inner sep=4pt,font=\sffamily,fill opacity=0.5] (SebastianoItem){#1)};}
}
\newcommand{\SebastianoHighlight}{\tikz[overlay,remember picture]{%
\fill[\SebastianoColor,fill opacity=0.5] ([yshift=4pt,xshift=-\pgflinewidth]SebastianoItem.east) -- ++(4pt,-4pt)
-- ++(-4pt,-4pt) -- cycle;
}}   
            %%%%%%%%%%%%%%%%%  Sect.        14 Oct 2024     %%%%%%%%%%%%%%%%%%%%%%%%%%%%%%%%%%%%%%%%%%%%%%%%%%%%%%%%%%%





%====================================================================

%====================================================================
\newcommand*{\authorimg}[1]%
    { \raisebox{-1\baselineskip}{\includegraphics[width=\imagesize]{#1}}}
\newlength\imagesize  

\usepackage{pgfplots}
\pgfplotsset{compat=1.17}

%==========================================================================================
\usepackage{libris} 
\usepackage[export]{adjustbox}% for positioning figures

\makeatletter
% Force le chapitre suivant sur la ligne succedant la fin du 
% chapitre précédent
\patchcmd{\chapter}{\if@openright\cleardoublepage\else\clearpage\fi}{}{}{}
\makeatother
\usepackage[Sonny]{fncychap}


%boîte de couleur grise
\tcbset{
  graybox/.style={
    colback=gray!20,
    colframe=black,
    sharp corners=downhill,
    boxrule=1pt,
    left=5pt,
    right=5pt,
    top=5pt,
    bottom=5pt,
    boxsep=0pt,
	 % <-- add four values for each corner
  }
}
\newtcolorbox{graybox}{graybox}

%==========================================================================================



\usepackage{xcolor}
\usepackage{varwidth}
\usepackage{varwidth}
%\usepackage{authblk}
\usepackage{nameref}
\usepackage{multicol,array}
\usepackage{tikz-cd}
\usepackage[ruled,vlined,linesnumbered]{algorithm2e}
\usepackage{comment} % enables the use of multi-line comments (\ifx \fi) 
\usepackage{import}
\usepackage{xifthen}
\usepackage{pdfpages}
\usepackage{transparent}


%\usepackage[french]{babel}
\usepackage{listings} % pour écrire du code dans un environnement
\lstset{
  basicstyle=\ttfamily,
  columns=fullflexible,
  keepspaces=true
}
\usepackage{float} % Pour forcer les images au bon endroit



\usepackage[T1]{fontenc}
\usepackage{csquotes}
%%%%%%%%%%%%%%%%%%%%%%%%%%%%%%%%%%%%%%%%%%%%%%%%%%%%%%%%%%%%%%%%%%%%%%%%%%%%%%%%%%%%%%%%%%%%%%%%%
%									ENSEMBLE DE COULEURS
%%%%%%%%%%%%%%%%%%%%%%%%%%%%%%%%%%%%%%%%%%%%%%%%%%%%%%%%%%%%%%%%%%%%%%%%%%%%%%%%%%%%%%%%%%%%%%%%%

\definecolor{myg}{RGB}{56, 140, 70}
\definecolor{myb}{RGB}{45, 111, 177}

\definecolor{mygbg}{RGB}{235, 253, 241}


\definecolor{myr}{RGB}{199, 68, 64}
\definecolor{mytheorembg}{HTML}{F2F2F9}
\definecolor{mytheoremfr}{HTML}{00007B}
\definecolor{mylenmabg}{HTML}{FFFAF8}
\definecolor{mylenmafr}{HTML}{983b0f}
\definecolor{mypropbg}{HTML}{f2fbfc}
\definecolor{mypropfr}{HTML}{191971}
\definecolor{myexamplebg}{HTML}{F2FBF8}
\definecolor{myexamplefr}{HTML}{88D6D1}
\definecolor{myexampleti}{HTML}{2A7F7F}
\definecolor{mydefinitbg}{HTML}{E5E5FF}
\definecolor{mydefinitfr}{HTML}{3F3FA3}
\definecolor{notesgreen}{RGB}{0,162,0}
\definecolor{myp}{RGB}{197, 92, 212}
\definecolor{mygr}{HTML}{2C3338}
\definecolor{myred}{RGB}{127,0,0}
\definecolor{myyellow}{RGB}{169,121,69}
\definecolor{myexercisebg}{HTML}{F2FBF8}
\definecolor{myexercisefg}{HTML}{88D6D1}
\definecolor{myred}{RGB}{127,0,0}
\definecolor{myyellow}{RGB}{169,121,69}

\definecolor{blue}{HTML}{008ED7}
\definecolor{mygray}{gray}{0.75}
\definecolor{lightBlue}{RGB}{235, 245, 255}
\definecolor{tcbcolred}{RGB}{255,0,0}
\definecolor{myGreen}{HTML}{009900}

% command to circle a text
\newtcbox{\entoure}[1][red]{on line,
	arc=3pt,colback=#1!10!white,colframe=#1!50!black,
	before upper={\rule[-3pt]{0pt}{10pt}},boxrule=1pt,
	boxsep=0pt,left=2pt,right=2pt,top=1pt,bottom=.5pt}
% command for the circle for the number of part entries
\newcommand\Circle[1]{\tikz[overlay,remember picture]
	\node[draw,circle, text width=18pt,line width=1pt] {#1};}

\newtcbox{\entouree}[1][red]{on line,
	arc=3pt,colback=#1!10!white,colframe=#1!50!white,
	before upper={\rule[-3pt]{0pt}{10pt}},boxrule=1pt,
	boxsep=0pt,left=2pt,right=2pt,top=1pt,bottom=.5pt}

\newcommand{\shellcmd}[1]{\\\indent\indent\texttt{\footnotesize\# #1}\\}

%=====================================================================

\patchcmd{\tableofcontents}{\contentsname}{\rmfamily\contentsname}{}{}
% patching of \@part to typeset the part number inside a framed box in its own line
% and to add color
\makeatletter
\patchcmd{\@part}
  {\addcontentsline{toc}{part}{\thepart\hspace{1em}#1}}
  {\addtocontents{toc}{\protect\addvspace{20pt}}
    \addcontentsline{toc}{part}{\huge{\protect\color{myyellow}%
      \setlength\fboxrule{2pt}\protect\Circle{%
        \hfil\thepart\hfil%
      }%
    }\\[2ex]\color{myred}\rmfamily#1}}{}{}

%\patchcmd{\@part}
%  {\addcontentsline{toc}{part}{\thepart\hspace{1em}#1}}
%  {\addtocontents{toc}{\protect\addvspace{20pt}}
%    \addcontentsline{toc}{part}{\huge{\protect\color{myyellow}%
%      \setlength\fboxrule{2pt}\protect\fbox{\protect\parbox[c][1em][c]{1.5em}{%
%        \hfil\thepart\hfil%
%      }}%
%    }\\[2ex]\color{myred}\sffamily#1}}{}{}
\makeatother
% this is the environment used to typeset the chapter entries in the ToC
% it is a modification of the leftbar environment of the framed package
\renewenvironment{leftbar}
  {\def\FrameCommand{\hspace{6em}%
    {\color{myyellow}\vrule width 2pt depth 6pt}\hspace{1em}}%
    \MakeFramed{\parshape 1 0cm \dimexpr\textwidth-6em\relax\FrameRestore}\vskip2pt%
  }
 {\endMakeFramed}

% using titletoc we redefine the ToC entries for parts, chapters, sections, and subsections
\titlecontents{part}
  [0em]{\centering}
  {\contentslabel}
  {}{}
\titlecontents{chapter}
  [0em]{\vspace*{2\baselineskip}}
  {\parbox{4.5em}{%
    \hfill\Huge\rmfamily\bfseries\color{myred}\thecontentspage}%
   \vspace*{-2.3\baselineskip}\leftbar\textsc{\small\chaptername~\thecontentslabel}\\\rmfamily}
  {}{\endleftbar}
\titlecontents{section}
  [8.4em]
  {\rmfamily\contentslabel{3em}}{}{}
  {\hspace{0.5em}\nobreak\color{myred}\normalfont\contentspage}
\titlecontents{subsection}
  [8.4em]
  {\rmfamily\contentslabel{3em}}{}{}  
  {\hspace{0.5em}\nobreak\color{myred}\contentspage}
%==========================================================================

%PYTHON LSTLISTING STYLE

% Define colors
\definecolor{Pgruvbox-bg}{HTML}{282828}
\definecolor{Pgruvbox-fg}{HTML}{ebdbb2}
\definecolor{Pgruvbox-red}{HTML}{fb4934}
\definecolor{Pgruvbox-green}{HTML}{b8bb26}
\definecolor{Pgruvbox-yellow}{HTML}{fabd2f}
\definecolor{Pgruvbox-blue}{HTML}{83a598}
\definecolor{Pgruvbox-purple}{HTML}{d3869b}
\definecolor{Pgruvbox-aqua}{HTML}{8ec07c}

% Define Python style
\lstdefinestyle{PythonGruvbox}{
	language=Python,
	identifierstyle=\color{lst-fg},
	basicstyle=\ttfamily\color{Pgruvbox-fg},
	keywordstyle=\color{Pgruvbox-yellow},
	keywordstyle=[2]\color{Pgruvbox-blue},
	stringstyle=\color{Pgruvbox-green},
	commentstyle=\color{Pgruvbox-aqua},
	backgroundcolor=\color{Pgruvbox-bg},
	%frame=tb,
	rulecolor=\color{Pgruvbox-fg},
	showstringspaces=false,
	keepspaces=true,
	captionpos=b,
	breaklines=true,
	tabsize=4,
	showspaces=false,
	numbers=left,
	numbersep=5pt,
	numberstyle=\tiny\color{gray},
	showtabs=false,
	columns=fullflexible,
	morekeywords={True,False,None},
	morekeywords=[2]{and,as,assert,break,class,continue,def,del,elif,else,except,exec,finally,for,from,global,if,import,in,is,lambda,nonlocal,not,or,pass,print,raise,return,try,while,with,yield},
	morecomment=[s]{"""}{"""},
	morecomment=[s]{'''}{'''},
	morecomment=[l]{\#},
	morestring=[b]",
	morestring=[b]',
	literate=
	{0}{{\textcolor{Pgruvbox-purple}{0}}}{1}
	{1}{{\textcolor{Pgruvbox-purple}{1}}}{1}
	{2}{{\textcolor{Pgruvbox-purple}{2}}}{1}
	{3}{{\textcolor{Pgruvbox-purple}{3}}}{1}
	{4}{{\textcolor{Pgruvbox-purple}{4}}}{1}
	{5}{{\textcolor{Pgruvbox-purple}{5}}}{1}
	{6}{{\textcolor{Pgruvbox-purple}{6}}}{1}
	{7}{{\textcolor{Pgruvbox-purple}{7}}}{1}
	{8}{{\textcolor{Pgruvbox-purple}{8}}}{1}
	{9}{{\textcolor{Pgruvbox-purple}{9}}}{1}
}
%====================================================================
% 
%====================================================================

% JAVA LSTLISTING STYLE IN Gruvbox Colorscheme
\definecolor{gruvbox-bg}{rgb}{0.282, 0.247, 0.204}
\definecolor{gruvbox-fg1}{rgb}{0.949, 0.898, 0.776}
\definecolor{gruvbox-fg2}{rgb}{0.871, 0.804, 0.671}
\definecolor{gruvbox-red}{rgb}{0.788, 0.255, 0.259}
\definecolor{gruvbox-green}{rgb}{0.518, 0.604, 0.239}
\definecolor{gruvbox-yellow}{rgb}{0.914, 0.808, 0.427}
\definecolor{gruvbox-blue}{rgb}{0.353, 0.510, 0.784}
\definecolor{gruvbox-purple}{rgb}{0.576, 0.412, 0.659}
\definecolor{gruvbox-aqua}{rgb}{0.459, 0.631, 0.737}
\definecolor{gruvbox-gray}{rgb}{0.518, 0.494, 0.471}

\definecolor{lst-bg}{RGB}{45, 45, 45}
\definecolor{lst-fg}{RGB}{220, 220, 204}
\definecolor{lst-keyword}{RGB}{215, 186, 125}
\definecolor{lst-comment}{RGB}{117, 113, 94}
\definecolor{lst-string}{RGB}{163, 190, 140}
\definecolor{lst-number}{RGB}{181, 206, 168}
\definecolor{lst-type}{RGB}{218, 142, 130}


\lstdefinestyle{JavaGruvbox}{
	language=Java,
	basicstyle=\ttfamily\color{lst-fg},
	keywordstyle=\color{lst-keyword},
	keywordstyle=[2]\color{lst-type},
	commentstyle=\itshape\color{lst-comment},
	stringstyle=\color{lst-string},
	numberstyle=\color{lst-number},
	backgroundcolor=\color{lst-bg},
	%frame=tb,
	rulecolor=\color{gruvbox-aqua},
	showstringspaces=false,
	keepspaces=true,
	captionpos=b,
	breaklines=true,
	tabsize=4,
	showspaces=false,
	showtabs=false,
	columns=fullflexible,
	morekeywords={var},
	morekeywords=[2]{boolean, byte, char, double, float, int, long, short, void},
	morecomment=[s]{/}{/},
	morecomment=[l]{//},
	morestring=[b]",
	morestring=[b]',
	numbers=left,
	numbersep=5pt,
	numberstyle=\tiny\color{gray},
}



%====================================================================
% 
%====================================================================


% Define Dracula color scheme for Java
\definecolor{draculawhite-background}{RGB}{237, 239, 252}
\definecolor{draculawhite-comment}{RGB}{98, 114, 164}
\definecolor{draculawhite-keyword}{RGB}{189, 147, 249}
\definecolor{draculawhite-string}{RGB}{152, 195, 121}
\definecolor{draculawhite-number}{RGB}{249, 189, 89}
\definecolor{draculawhite-operator}{RGB}{248, 248, 242}

% Define JavaDraculaWhite lstlisting environment
\lstdefinestyle{JavaDraculaWhite}{
    language=Java,
    backgroundcolor=\color{draculawhite-background},
    commentstyle=\itshape\color{draculawhite-comment},
    keywordstyle=\color{draculawhite-keyword},
    stringstyle=\color{draculawhite-string},
    basicstyle=\ttfamily\footnotesize\color{black},
    identifierstyle=\color{black},
    keywordstyle=\color{draculawhite-keyword}\bfseries,
    morecomment=[s][\color{draculawhite-comment}]{/**}{*/},
    showstringspaces=false,
    showspaces=false,
    breaklines=true,
    frame=single,
    rulecolor=\color{draculawhite-operator},
    tabsize=2,  
	numbers=left,
	numbersep=4pt,
	numberstyle=\ttfamily\tiny\color{gray}
}
%====================================================================
% 
%====================================================================
% Define PythonDraculaWhite lstlisting environment 
\definecolor{draculawhite-bg}{HTML}{FAFAFA}
\definecolor{draculawhite-fg}{HTML}{282A36}
\definecolor{pdraculawhite-keyword}{HTML}{BD93F9}

\definecolor{pdraculawhite-comment}{HTML}{6272A4}
\definecolor{draculawhite-number}{HTML}{FF79C6}


\lstdefinestyle{PythonDraculaWhite}{
    language=Python,
    basicstyle=\ttfamily\small\color{draculawhite-fg},
    backgroundcolor=\color{draculawhite-background},
    keywordstyle=\color{orange}\bfseries,
    stringstyle=\color{draculawhite-string},
    commentstyle=\color{pdraculawhite-comment}\itshape,
    numberstyle=\color{draculawhite-number},
    showstringspaces=false,
	showspaces=false,
    breaklines=true,
	frame=single,
	rulecolor=\color{draculawhite-operator}, 
    tabsize=4,
    morekeywords={as,with,1,2,3,4, 5,6,7,8,9,True,False},
    %escapeinside={(*@}{@*)},
    numbers=left,
    numbersep=5pt,
    %xleftmargin=15pt,
    %framexleftmargin=15pt,
    %framexrightmargin=0pt,
    %framexbottommargin=0pt,
    %framextopmargin=0pt,
    %rulecolor=\color{draculawhite-fg},
    %frame=tb,
    %aboveskip=0pt,
    %belowskip=0pt,
    %captionpos=b,
	numberstyle=\ttfamily\tiny\color{gray} 
}
%====================================================================
% 
%====================================================================

% Define colors for HTML langage
\definecolor{html-orange}{HTML}{FF5733}
\definecolor{html-yellow}{HTML}{F0E130}
\definecolor{html-green}{HTML}{50FA7B}
\definecolor{html-blue}{HTML}{5AFBFF}
\definecolor{html-purple}{HTML}{BD93F9}
\definecolor{html-pink}{HTML}{FF80BF}
\definecolor{html-gray}{HTML}{6272A4}
\definecolor{html-white}{HTML}{F8F8F2}

% Defines a new HTML5 langage that extend on the html langange
\lstdefinestyle{HTMLDraculaWhite}{
  language=HTML,
  backgroundcolor=\color{html-white},
  basicstyle=\ttfamily\color{html-gray},
  keywordstyle=\color{html-blue},
  stringstyle=\color{html-orange},
  commentstyle=\color{html-green},
  tagstyle=\color{html-yellow},
  moredelim=[s][\color{html-pink}]{<!--}{-->},
  moredelim=[s][\color{html-purple}]{\{}{\}},
  showstringspaces=false,
  tabsize=2,
  breaklines=true,
  columns=fullflexible,
  %frame=single,
  framexleftmargin=5mm,
  xleftmargin=10mm,
  numbers=left,
  numberstyle=\tiny\color{html-gray},
  escapeinside={<@}{@>}
}

%====================================================================
% 
%====================================================================
% Define the colors needed for the HTMLDraculaDark environment
\definecolor{htmltag}{HTML}{ff79c6}
\definecolor{htmlattr}{HTML}{f1fa8c}
\definecolor{htmlvalue}{HTML}{bd93f9}
\definecolor{htmlcomment}{HTML}{6272a4}
%\definecolor{htmltext}{HTML}{f8f8f2}
\definecolor{htmltext}{HTML}{401E31}
\definecolor{htmlbackground}{HTML}{282a36}
\definecolor{comphtmlbackground}{HTML}{8093FF}
%\definecolor{htmlbackground}{HTML}{4D5169}

% Define the HTMLDraculaDark environment
\lstdefinestyle{HTMLDraculaDark}{
    basicstyle=\bfseries\ttfamily\color{htmltext},
    commentstyle=\itshape\color{htmlcomment},
    keywordstyle=\bfseries\color{htmltag},
    stringstyle=\color{htmlvalue},
    emph={DOCTYPE,html,head,body,div,span,a,script},
    emphstyle={\color{htmltag}\bfseries},
    sensitive=true,
    showstringspaces=false,
    backgroundcolor=\color{white},
    %frame=tb,
    language=HTML,
    tabsize=4,
    breaklines=true,
    breakatwhitespace=true,
    numbers=left,
    numbersep=10pt,
    numberstyle=\small\bfseries\ttfamily\color{htmlcomment},
    escapeinside={<@}{@>},
	rulecolor=\color{htmlbackground},
	xleftmargin=20pt,
	% Add a vertical line for opening and closing tags
    %frame=lines,
    framesep=2pt,
    framexleftmargin=4pt,
    % Add a vertical line for closing tags that go through multiple lines
    breaklines=true,
    postbreak=\mbox{\textcolor{gray}{$\hookrightarrow$}\space},
    showlines=true,
	% Add a rule to apply commentstyle to keywords inside comments
    moredelim=[s][\itshape\color{htmlcomment}]{<!--}{-->},
    morekeywords={id,class,type,name,value,placeholder,checked,src,href,alt}
}




%====================================================================
% 
%====================================================================






% Crée un environnement "Theorem" numéroté en fonction du document
\tcbuselibrary{theorems,skins,hooks} 
\newtcbtheorem{Theorem}{Théorème}
{%
	enhanced,
	breakable,
	colback = mytheorembg,
	frame hidden,
	boxrule = 0sp,
	borderline west = {2pt}{0pt}{mytheoremfr},
	sharp corners,
	detach title,
	before upper = \tcbtitle\par\smallskip,
	coltitle = mytheoremfr,
	fonttitle = \bfseries\fontfamily{lmss}\selectfont,
	description font = \mdseries\fontfamily{lmss}\selectfont,
	separator sign none,
	segmentation style={solid, mytheoremfr},
}
{thm}

% Crée un environnement "Preuve" numéroté en fonction du document
\tcbuselibrary{theorems,skins,hooks}
\newtcbtheorem{Preuve}{Preuve}
{
	enhanced,
	breakable,
	colback=white,
	frame hidden,
	boxrule = 0sp,
	borderline west = {2pt}{0pt}{mytheoremfr},
	sharp corners,
	detach title,
	before upper = \tcbtitle\par\smallskip,
	coltitle = mytheoremfr,
	description font=\fontfamily{lmss}\selectfont,
	fonttitle=\fontfamily{lmss}\selectfont\bfseries,
	separator sign none,
	segmentation style={solid, mytheoremfr},
}
{th}


% Crée un environnement "Preuve" numéroté en fonction du document
\tcbuselibrary{theorems,skins,hooks}
\newtcbtheorem{Explication}{Explication}
{
	enhanced,
	breakable,
	colback=white,
	frame hidden,
	boxrule = 0sp,
	borderline west = {2pt}{0pt}{mytheoremfr},
	sharp corners,
	detach title,
	before upper = \tcbtitle\par\smallskip,
	coltitle = mytheoremfr,
	description font=\fontfamily{lmss}\selectfont,
	fonttitle=\fontfamily{lmss}\selectfont\bfseries,
	separator sign none,
	segmentation style={solid, mytheoremfr},
}
{th}




% Crée un environnement "Example" numéroté en fonction du document
\tcbuselibrary{theorems,skins,hooks}
\newtcbtheorem{Example}{Exemple.}
{
	enhanced,
	breakable,
	colback=lightBlue,
	frame hidden,
	boxrule = 0sp,
	borderline west = {2pt}{0pt}{myb},
	sharp corners,
	detach title,
	before upper = \tcbtitle\par\smallskip,
	coltitle = myb,
	description font=\fontfamily{lmss}\selectfont,
	fonttitle=\fontfamily{lmss}\selectfont\bfseries,
	separator sign none,
	segmentation style={solid, mytheoremfr},
}
{th}



% Crée un environnement "EExample" numéroté en fonction du document
\tcbuselibrary{theorems,skins,hooks}
\newtcbtheorem{EExample}{Exemple}
{
	enhanced,
	breakable,
	colback=white,
	frame hidden,
	boxrule = 0sp,
	borderline west = {2pt}{0pt}{myb},
	sharp corners,
	detach title,
	before upper = \tcbtitle\par\smallskip,
	coltitle = myb,
	description font=\mdseries\fontfamily{lmss}\selectfont,
	fonttitle=\fontfamily{lmss}\selectfont\bfseries,
	separator sign none,
	segmentation style={solid, mytheoremfr},
}
{th}



% Crée un environnement "Lemme" numéroté en fonction du document
\tcbuselibrary{theorems,skins,hooks}
\newtcbtheorem{Lemme}{Lemme}
{
	enhanced,
	breakable,
	colback=mylenmabg,
	frame hidden,
	boxrule = 0sp,
	borderline west = {2pt}{0pt}{mylenmafr},
	sharp corners,
	detach title,
	before upper = \tcbtitle\par\smallskip,
	coltitle = mylenmafr,
	description font=\mdseries\fontfamily{lmss}\selectfont,
	fonttitle=\fontfamily{lmss}\selectfont\bfseries,
	separator sign none,
	segmentation style={solid, mytheoremfr},
}
{th}


\tcbuselibrary{theorems,skins,hooks}
\newtcbtheorem{PreuveL}{Preuve.}
{
	enhanced,
	breakable,
	colback=white,
	frame hidden,
	boxrule = 0sp,
	borderline west = {2pt}{0pt}{mylenmafr},
	sharp corners,
	detach title,
	before upper = \tcbtitle\par\smallskip,
	coltitle = mylenmafr,
	description font=\fontfamily{lmss}\selectfont,
	fonttitle=\fontfamily{lmss}\selectfont\bfseries,
	separator sign none,
	segmentation style={solid, mytheoremfr},
}
{th}


\newtcbtheorem{Remarque}{Remarque}
{
	enhanced,
	breakable,
	colback=white,
	frame hidden,
	boxrule = 0sp,
	borderline west = {2pt}{0pt}{myb},
	sharp corners,
	detach title,
	before upper = \tcbtitle\par\smallskip,
	coltitle = myb,
	description font=\mdseries\fontfamily{lmss}\selectfont,
	fonttitle=\fontfamily{lmss}\selectfont\bfseries,
	separator sign none,
	segmentation style={solid, mytheoremfr},
}
{th}


\newtcbtheorem{DefG}{Définition}
{
	enhanced,
	breakable,
	colback=mygbg,
	frame hidden,
	boxrule = 0sp,
	borderline west = {2pt}{0pt}{myg},
	sharp corners,
	detach title,
	before upper = \tcbtitle\par\smallskip,
	coltitle = myg,
	description font=\mdseries\fontfamily{lmss}\selectfont,
	fonttitle=\fontfamily{lmss}\selectfont\bfseries,
	separator sign none,
	segmentation style={solid, mytheoremfr},
}
{th}



% Crée une boîte ayant la même couleur que l'environnement theorem.
\tcbuselibrary{theorems,skins,hooks}
\newtcolorbox{Theoremcon}
{%
	enhanced
	,breakable
	,colback = mytheorembg
	,frame hidden
	,boxrule = 0sp
	,borderline west = {2pt}{0pt}{mytheoremfr}
	,sharp corners
	,description font = \mdseries
	,separator sign none
}

% Crée un environnement "Definition" numéroté en fonction de la section
\newtcbtheorem[number within=chapter]{Definition}{Définition}{enhanced,
	before skip=2mm,after skip=2mm, colback=red!5,colframe=red!80!black,boxrule=0.5mm,
	attach boxed title to top left={xshift=1cm,yshift*=1mm-\tcboxedtitleheight}, varwidth boxed title*=-3cm,
	boxed title style={frame code={
			\path[fill=tcbcolback!10!red]
			([yshift=-1mm,xshift=-1mm]frame.north west)
			arc[start angle=0,end angle=180,radius=1mm]
			([yshift=-1mm,xshift=1mm]frame.north east)
			arc[start angle=180,end angle=0,radius=1mm];
			\path[left color=tcbcolback!10!myred,right color=tcbcolback!10!myred,
			middle color=tcbcolback!60!myred]
			([xshift=-2mm]frame.north west) -- ([xshift=2mm]frame.north east)
			[rounded corners=1mm]-- ([xshift=1mm,yshift=-1mm]frame.north east)
			-- (frame.south east) -- (frame.south west)
			-- ([xshift=-1mm,yshift=-1mm]frame.north west)
			[sharp corners]-- cycle;
		},interior engine=empty,
	},
	fonttitle=\bfseries,
	title={#2},#1}{def}

% Crée un environnement "definition" numéroté en fonction du Chapitre
\newtcbtheorem[number within=section]{definition}{Définition}{enhanced,
	before skip=2mm,after skip=2mm, colback=red!5,colframe=red!80!black,boxrule=0.5mm,
	attach boxed title to top left={xshift=1cm,yshift*=1mm-\tcboxedtitleheight}, varwidth boxed title*=-3cm,
	boxed title style={frame code={
			\path[fill=tcbcolback]
			([yshift=-1mm,xshift=-1mm]frame.north west)
			arc[start angle=0,end angle=180,radius=1mm]
			([yshift=-1mm,xshift=1mm]frame.north east)
			arc[start angle=180,end angle=0,radius=1mm];
			\path[left color=tcbcolback!60!black,right color=tcbcolback!60!black,
			middle color=tcbcolback!80!black]
			([xshift=-2mm]frame.north west) -- ([xshift=2mm]frame.north east)
			[rounded corners=1mm]-- ([xshift=1mm,yshift=-1mm]frame.north east)
			-- (frame.south east) -- (frame.south west)
			-- ([xshift=-1mm,yshift=-1mm]frame.north west)
			[sharp corners]-- cycle;
		},interior engine=empty,
	},
	fonttitle=\bfseries,
	title={#2},#1}{def}

\usetikzlibrary{arrows,calc,shadows.blur}
\tcbuselibrary{skins}
\newtcolorbox{note}[1][]{%
	enhanced jigsaw,
	colback=gray!20!white,%
	colframe=gray!80!black,
	size=small,
	boxrule=1pt,
	title=\textbf{Note : },
	halign title=flush center,
	coltitle=black,
	breakable,
	drop shadow=black!50!white,
	attach boxed title to top left={xshift=1cm,yshift=-\tcboxedtitleheight/2,yshifttext=-\tcboxedtitleheight/2},
	minipage boxed title=1.5cm,
	boxed title style={%
		colback=white,
		size=fbox,
		boxrule=1pt,
		boxsep=2pt,
		underlay={%
			\coordinate (dotA) at ($(interior.west) + (-0.5pt,0)$);
			\coordinate (dotB) at ($(interior.east) + (0.5pt,0)$);
			\begin{scope}
				\clip (interior.north west) rectangle ([xshift=3ex]interior.east);
				\filldraw [white, blur shadow={shadow opacity=60, shadow yshift=-.75ex}, rounded corners=2pt] (interior.north west) rectangle (interior.south east);
			\end{scope}
			\begin{scope}[gray!80!black]
				\fill (dotA) circle (2pt);
				\fill (dotB) circle (2pt);
			\end{scope}
		},
	},
	#1,
}


% Crée un environnement "qstion" 
\newtcbtheorem{qstion}{Question}{enhanced,
	breakable,
	colback=white,
	colframe=mygr,
	attach boxed title to top left={yshift*=-\tcboxedtitleheight},
	fonttitle=\bfseries,
	title={#2},
	boxed title size=title,
	boxed title style={%
		sharp corners,
		rounded corners=northwest,
		colback=tcbcolframe,
		boxrule=0pt,
	},
}{def}


% Pour créer un environnement "Liste" 

\tcbuselibrary{theorems,skins,hooks}
\newtcbtheorem[number within=section]{Liste}{Liste}
{%
	enhanced
	,breakable
	,colback = myp!10
	,frame hidden
	,boxrule = 0sp
	,borderline west = {2pt}{0pt}{myp!85!black}
	,sharp corners
	,detach title
	,before upper = \tcbtitle\par\smallskip
	,coltitle = myp!85!black
	,fonttitle = \bfseries\sffamily
	,description font = \mdseries
	,separator sign none
	,segmentation style={solid, myp!85!black}
}
{th}


\tcbuselibrary{theorems,skins,hooks}
\newtcbtheorem{Syntaxe}{Syntaxe.}
{%
	enhanced
	,breakable
	,colback = myp!10
	,frame hidden
	,boxrule = 0sp
	,borderline west = {2pt}{0pt}{myp!85!black}
	,sharp corners
	,detach title
	,before upper = \tcbtitle\par\smallskip
	,coltitle = myp!85!black
	,fonttitle = \bfseries\fontfamily{lmss}\selectfont 
	,description font = \mdseries\fontfamily{lmss}\selectfont 
	,separator sign none
	,segmentation style={solid, myp!85!black}
}
{th}



% Crée un environnement "Concept" numéroté en fonction du document
\tcbuselibrary{theorems,skins,hooks}
\newtcbtheorem{Concept}{Concept.}
{
	enhanced,
	breakable,
	colback=mylenmabg,
	frame hidden,
	boxrule = 0sp,
	borderline west = {2pt}{0pt}{mylenmafr},
	sharp corners,
	detach title,
	before upper = \tcbtitle\par\smallskip,
	coltitle = mylenmafr,
	description font=\mdseries\fontfamily{lmss}\selectfont,
	fonttitle=\fontfamily{lmss}\selectfont\bfseries,
	separator sign none,
	segmentation style={solid, mytheoremfr},
}
{th}


% Crée un environnement "codeEx" numéroté en fonction du document
\tcbuselibrary{theorems,skins,hooks}
\newtcbtheorem{codeEx}{Exemple}
{
	enhanced,
	breakable,
	colback=white,
	frame hidden,
	boxrule = 0sp,
	borderline west = {2pt}{0pt}{gruvbox-bg},
	sharp corners,
	detach title,
	before upper = \tcbtitle\par\smallskip,
	coltitle = gruvbox-bg,
	description font=\md:wqseries\fontfamily{lmss}\selectfont,
	fonttitle=\fontfamily{lmss}\selectfont\bfseries,
	separator sign none,
	segmentation style={solid, mytheoremfr},
}
{th}


% Crée un environnement "codeEx" numéroté en fonction du document
\tcbuselibrary{theorems,skins,hooks}
\newtcbtheorem{codeRem}{Remarque.}
{
	enhanced,
	breakable,
	colback=white,
	frame hidden,
	boxrule = 0sp,
	borderline west = {2pt}{0pt}{gruvbox-bg},
	sharp corners,
	detach title,
	before upper = \tcbtitle\par\smallskip,
	coltitle = gruvbox-bg,
	description font=\mdseries\fontfamily{lmss}\selectfont,
	fonttitle=\fontfamily{lmss}\selectfont\bfseries,
	separator sign none,
	segmentation style={solid, mytheoremfr},
}
{th}


\tcbuselibrary{theorems,skins,hooks}
\newtcbtheorem{Identite}{Identité.}
{
	enhanced,
	breakable,
	colback=white,
  before upper=\tcbtitle\par\Hugeskip,
	frame hidden,
	boxrule = 0sp,
	borderline west = {2pt}{0pt}{gruvbox-bg},
	sharp corners,
	detach title,
	before upper = \tcbtitle\par\smallskip,
	coltitle = gruvbox-bg,
	description font=\mdseries\fontfamily{lmss}\selectfont,
	fonttitle=\fontfamily{lmss}\selectfont\bfseries,
	fontlower=\fontfamily{cmr}\selectfont,
  separator sign none,
	segmentation style={solid, mytheoremfr},
}
{th}

\tcbuselibrary{theorems,skins,hooks}
\newtcbtheorem{Exercice}{Exercice}
{
	enhanced,
	breakable,
	colback=white,
  before upper=\tcbtitle\par\Hugeskip,
	frame hidden,
	boxrule = 0sp,
	borderline west = {2pt}{0pt}{gruvbox-green},
	sharp corners,
	detach title,
	before upper = \tcbtitle\par\smallskip,
	coltitle = gruvbox-green,
	description font=\mdseries\fontfamily{lmss}\selectfont,
	fonttitle=\fontfamily{lmss}\selectfont\bfseries,
	fontlower=\fontfamily{cmr}\selectfont,
  separator sign none,
	segmentation style={solid, mytheoremfr},
}
{th}

% Crée un environnement "Réponse" numéroté en fonction du document
\tcbuselibrary{theorems,skins,hooks}
\newtcbtheorem{Reponse}{Réponse}
{
	enhanced,
	breakable,
	colback=white,
	frame hidden,
	boxrule = 0sp,
	borderline west = {2pt}{0pt}{mytheoremfr},
	sharp corners,
	detach title,
	before upper = \tcbtitle\par\smallskip,
	coltitle = mytheoremfr,
	description font=\fontfamily{lmss}\selectfont,
	fonttitle=\fontfamily{lmss}\selectfont\bfseries,
	separator sign none,
	segmentation style={solid, mytheoremfr},
}
{th}

\newtcbtheorem{Definitionx}{Définition}
{
enhanced,
breakable,
colback=red!5,
  before upper=\tcbtitle\par\Hugeskip,
frame hidden,
boxrule = 0sp,
borderline west = {2pt}{0pt}{red!80!black},
sharp corners,
detach title,
before upper = \tcbtitle\par\smallskip,
coltitle = red!80!black,
description font=\mdseries\fontfamily{lmss}\selectfont,
fonttitle=\fontfamily{lmss}\selectfont\bfseries,
fontlower=\fontfamily{cmr}\selectfont,
  separator sign none,
segmentation style={solid, mytheoremfr},
}
{th}



\usepackage[scr]{rsfso}


\title{\Huge{Structure Discrète}\\{IFT1065}\\{\textbf{Devoir 2}} \\ {Récursivité et Preuves}}
\author{\huge{Franz Girardin et Aiya}}
\date{\today}
\lstset{inputencoding=utf8/latin1}

            %%%%%%%%%%%%%%%%%  Sect.                          %%%%%%%%%%%%%%%%%%%%%%%%%%%%%%%%%%%%%%%%%%%%%%%%%%%%%%%%%
\usepackage{helvet}
\titleformat{\chapter}
  {\fontfamily{phv}\bfseries\huge} % format
  {}                % label
  {0pt}             % sep
  {\color{myb}\huge}           % before-code



\titleformat{\section}
  {\normalfont\scshape}{\thesection}{1em}{}


% Customizing the spacing for the chapter titles
\titlespacing*{\chapter}{0pt}{0pt}{20pt}

\usepackage{mathpazo}
\begin{document}
\maketitle
\pagebreak
\tableofcontents 
\pagebreak

\pagebreak
\begin{multicols*}{2}


    \chapter{Résolution de problèmes}

    \section*{Problème 1 $\quad$ $\cdot$  $\quad$ Divisibilité}
    \begin{enumerate}
        \item Montrez que $a$ divise $b$ si et seulement si $an$ divise $bn$. Reformulez la proposition
        en langage logique, puis écrivez sa preuve en explicitant chaque technique utilisée.
    \end{enumerate}

    Soit le proposition $P(a, b, n) : a $ divise $b$ \textbf{si et seulement si} $an$ divise $bn$, 
    nous pouvons réécrire $P(a, b, n)$ en language logique de la façon suivante :
    \[ P(a,b,n) \Coloneqq a|b \Leftrightarrow  an|bn \]

    Nous savons qu'une telle proposition biconditionnelle est une synthèse de \textbf{propositions conditionnelles}
    distinctes que nous appellerons $P_r$ et $P_s$ :

    \begin{align}
              &P_q \Coloneqq a|b \implies  an|bn \\ 
              &P_r \Coloneqq an|bn \implies a|b
    \end{align}     
    \paragraph{}
    \textbf{Pour montrer la véracité de $P(a,b,c)$}, nous allons donc montrer que $P_q$ et $P_r$ sont vrais. 
    
    \begin{prop}{($P_q$)}{}
        \[ a|b \implies  an|bn \]
    \end{prop}
    
    \begin{Preuve*}{($P_q$)}{}
        Nous allons montrer $P_q$ par \textcolor{red}{\textit{preuve directe}}.    
        Supposons que $a$ divise $b$. \textbf{Par définition}, cela signifie qu'il existe un nombre 
        $c \in \mathbb{Z}$ tel que 
        \textcolor{myb}{$b = ac$}. Et, trivialement, $bn = acn$ Nous avons alors : 
        \begin{align*}
            an|bn &\equiv an|\textcolor{myg}{(ac) \cdot n} \\
                        &\equiv an|(an) \cdot c \\
                        &\equiv an|an \cdot c
        \end{align*}
        Autrement dit, si $an$ divise \textcolor{myg}{$(an)\cdot  c$}, 
        cela veut dire qu'il existe un nombre $d \in \mathbb{Z}$ tel que 
        $an \cdot d = an \cdot c$. D'une part, cela implique  que $d = c$. 
        Par ailleurs, nous constatons que $bn$ est un multiple de $an$, car $bn$ peut être exprimé comme 
        $an$ multiplié par un entier $c$. Par conséquent, nous concluons 
        que $an$ divise $bn$.
        \qed 
    \end{Preuve*}    

    \begin{prop}{($P_r$)}{}
        \[ an|bn \implies  a|b \]
    \end{prop}

    \begin{Preuve*}{($P_r$)}{}
        Nous voulons prouver que si $an$ divise $bn$, \textbf{alors} $a$ divise forcément $b$ ($P_r$). 
        Nous allons montrer $P_r$ par \textcolor{red}{\textit{preuve directe}}. Supposons que $an$ divise $bn$. 
        \textbf{Par définition}, cela signifie qu'il existe un entier $k \in \mathbb{Z}$ tel que 
        \textcolor{myb}{$an \cdot k = bn$}.  En divisant les deux côtés de l'équation par $n$, 
        nous obtenons $b = ak$. \textbf{Or}, si nous substituons la valeur que 
        nous venons de dériver de $b$, nous avons : 
                            \[ a|b \equiv a|ak \]
        Cette équivalence tient, puisque toute division de $ak$ par $a$ implique de diviser $b$ par $a$.  
        Autrement dit, $ak$ est une multiple de 
        $a$; on peut obtenir $ak$ en multipliant $a$ par un facteur $k$. Cela revient à dire 
        que $b$ est un multiple de $a$ et donc, \textbf{par définition}, $a|b$ Par conséquent, nous 
        concluons que si $an|bn$, alors $a|b$, puisque $b$ est un multiple de $a$. \qed
    \end{Preuve*}
    

    \paragraph{}
    Nous venons de montrer que la proposition $P_q$ et sa réciproque $P_r$ sont toutes deux vraies 
    Par la définition d'une \textit{proposition biconditionnelle}, nous concluons que la proposition :
                        \[ an|bn \Leftrightarrow  a|b \]
    est vraie. Autrement dit, $P(a,b,n) \Coloneqq$ \textit{an divise bn si et seulement si a divise b }
    est vraie. \qed
    

    \begin{enumerate}
        \item[2.] Montrez que si n ne divise pas ab, alors n ne divise ni a, ni b.
            Reformulez la proposition en langage logique, puis écrivez sa preuve en explicitant
            chaque technique utilisée.
    \end{enumerate}

    Soit la proposition $P^{\prime}(a,b, n)$ : si $n$ ne divise pas $ab$, alors $n$ ne divise 
    ni $a$, ni $b$. Nous pouvons réécrire $P^{\prime}(a,b, n)$ en language logique de la façon suivante : 
                        \[ P^{\prime}(a,b, n) \Coloneqq  n \nmid ab \implies  (n \nmid a) \land (n \nmid b) \]
    Nous faisons face à une proposition conditionnelle où le côté droit de l'implication contient 
    une conjonction.

    \begin{prop}{($P^{\prime}(a,b, n)$)}{}
        \[ n \nmid ab \implies  (n \nmid a) \land (n \nmid b) \]
    \end{prop}

    \begin{Preuve*}{}{}
        Nous voulons prouver que si un nombre $n$ ne divise pas $ab$, alors ce nombre ne divise ni $a$ ni $b$. 
        Nous allons montrer $P^{\prime}(a,b, n)$ par \textcolor{red}{\textit{contraposé}}. Supposon la négation 
        du côté droit de l'implication. Autrement dit, supposons :
        \[ \neg \left( (n \nmid a) \land (n \nmid b) \right) \]
        Par \textbf{De Morgan}, nous avons 

        \begin{align*}
            \neg \left( (n \nmid a) \land (n \nmid b) \right) &\equiv  \neg (n \nmid a) \lor \neg (n \nmid b)
                    \\ 
                                  &\equiv  (n | a) \lor (n | b)
        \end{align*}
        Nous allons alors prouver que si $n|a$ \textbf{ou} $n|b$, alors, $n|ab$, soit la \textbf{contraposée} de 
        $P^{\prime}(a,b, n)$. 

        \begin{align}
                (n|a) \lor (n|b) \implies n|ab                    
        \end{align}

        \begin{note}{}{}
            Intuitivement, nous savons déjà que si on nombre $n$ divise un nombre
            $a$ ou un nombre $b$, ce nombre $n$ divise nécessairement, le produit $ab$.
        \end{note}

        \textit{\textcolor{red}{Preuve par cas}}. \vspace{1em} \\
        \underline{\textbf{Cas 1}} : $n|a$ \underline{\textbf{Cas 2}} $n|b$. \\ 
        Sans perte de généralité, si $n|a$, \textbf{alors} il existe un entier $k \in \mathbb{Z}$ tel que 
        $nk = a$. Donc, $ab = (nk) \cdot b$. Et pour diviser $ab$, 
        il faut que $n$ divise $n \cdot (kb)$ ;  autrement dit, \textbf{pour que $n$ divise $ab$}, 
        \textbf{il suffit que que $n$ divise $n$}, ce qui est toujours vrai pour tous $n \in \mathbb{Z^*}$. 
        \begin{align*}
            n|ab &\equiv n|(nk) \cdot b \\
                 &\equiv n|n \cdot (kb) \\ 
                 &\equiv n|nkb
        \end{align*}
        Par conséquent, $nkb$ est un multiple de $n$ et nous concluons alors que $n$ divise $ab$, puisque 
        par substitutions $n$ divise $ab$.
        
        \paragraph{}
        Ayant, indiqué que le \underline{\textbf{Cas 2}} se traite de façon similaire au \underline{\textbf{Cas 1}},
        nous concluons que, dans les deux cas, $n$ divise $ab$. Nous venons donc de prouver la contraposée 
        de $P^{\prime}(a,b,n)$. Puisque la contraposée de $P^{\prime}(a,b,n)$ est vraie, il s'ensuit que 
        $P^{\prime}(a,b,n)$ est aussi vraie. Nous concluons alors que si un entier $n$ ne divise pas un produit 
        $ab$, alors cet entier $n$ ne divise ni $a$ ni $b$. \qed 
    \end{Preuve*}

    \begin{enumerate}
        \item[3.] Remarquez que la réciproque de (2.) n’est pas vraie. Donnez un contre-exemple. 
    \end{enumerate}

    La récirpoque de $P^{\prime}(a,b, n)$, $Q^{\prime}(a,b, n)$ peut être réécrite comme suit: 
    \[  Q^{\prime}(a,b, n) \Coloneqq (n \nmid a) \land (n \nmid b)  \implies  n \nmid ab \]
    Et cela revient à affirmer que \textit{ si un nombre \textcolor{myb}{$n$} ne divise   
        pas un nombre \textcolor{myb}{$a$} ni un nombre \textcolor{myb}{$b$}, alors ce nombre
    \textcolor{myb}{$n$} ne divise pas le produit \textcolor{myb}{$ab$}}. Cette proposition est fausse.   

    \begin{Preuve*}{}{}
        Nous allons prouver que la réciproque de $P^{\prime}(a,b,n)$ est fausse par 
        \textcolor{red}{\textit{contre-exemple}}. 
        Pour réfuter $Q^{\prime}(a,b,n)$, nous allons montrer qu'il existe des entiers 
        $a, b, n \in \mathbb{Z}$ tels que $n \nmid a$ et $n \nmid b$ et pourtant $n | ab$. 
        Soit $n = 4$, $a = 2$, $b = 6$, et $ab = 12$. Nous savons que 
        $4$ ne divise pas $2$. Nous savons également que $4$ ne divise $6$. Or, $4$ divise 
        $12$. Nous avons donc un exemple de $a, b, n \in \mathbb{Z}$ qui contredit $Q^{\prime}(a,b,n)$. 
        Nous concluons que $Q^{\prime}(a,b,n)$ est faux. \qed
    \end{Preuve*}

    \begin{enumerate}
        \item[4.] Montrez que n divise a et b si et seulement si n divise pgcd(a; b). Reformulez
        la proposition en langage logique, puis écrivez sa preuve en explicitant chaque
        technique utilisée.
    \end{enumerate}

    Soit la proposition $P^{\prime\prime}(a,b,n)$ : $n$ divise $a$ \textbf{et} $b$  
    \textbf{si et seulement si}  $n$ divise $pgcd(a,b)$, nous pouvons réécrire $P^{\prime\prime}(a,b,n)$ 
    en language logique de la façon suivante: 
    \[ P^{\prime\prime}(a,b,n)  \Coloneqq (n|a) \land (n|b)  \Leftrightarrow n|pgcd(a,b) \]
    Nous savons qu'une telle proposition biconditionnelle est une synthèse de \textbf{propositions conditionnelles}
    distinctes que nous appellerons $P^{\prime\prime}_r$ et $P^{\prime\prime}_s$ :


    \begin{align}
              &P^{\prime\prime}_q \Coloneqq (n|a) \land (n|b) \implies  n|pgcd(a,b) \\ 
              &P^{\prime\prime}_r \Coloneqq n|pgcd(a,b) \implies (n|a) \land (n|b)
    \end{align}  
    \paragraph{}
    \textbf{Pour montrer la véracité de $P^{\prime\prime}(a,b,c)$}, 
    nous allons donc montrer que $P^{\prime\prime}_q$ et $P^{\prime\prime}_r$ sont vrais.


    \begin{prop}{($P^{\prime\prime}_q(a,b, n)$)}{}
        \[ (n|a) \land (n|b) \implies  n|pgcd(a,b) \]
    \end{prop}

    \begin{Preuve*}{}{}
       Nous voulons montrer que si $n$ divise $a$ et $n$ divise $b$, \textbf{alors}, $n$ 
       divise le plus grand commun diviseur de $a$ et $b$. Nous allons montrer 
        $P^{\prime\prime}_q(a,b, n)$ par \textcolor{red}{\textit{preuve directe}}. 
        \begin{Lemme}{}{}
            Le pgcd(a,b) est un multiple de n'importe quel diviseur commun de $a$ et $b$.
        \end{Lemme}
        Ce Lemme découle de la définition du pgcd, qui est le plus grand diviseur commun de \( a \) et \( b \),
        impliquant qu'il est un multiple de tous les autres diviseurs communs. \vspace{1em}\\

        Supposons que $n$ divise $a$ et $n$ divise $b$. \textbf{Par définition}, $n$ est un diviseur 
        commun de $a$ et $b$:  
        \[ n \Coloneqq dc(a,b) \]
        \textbf{Or}, si $n$ est un diviseur commun de $a$ et $b$, \textbf{il faut} que $n$ divise le 
        plus grand diviseur commun de $a$ et $b$, par le \textbf{\textcolor{brown}{Lemme 1}}.
        En effet, si $n$ est bien un diviseur commun de $a$ et $b$, il y a deux cas possibles. Soit :
        \begin{itemize}
            \item $n$ est l'unique diviseur commun de $a$ et $b$ et donc n est est le plus grand commun diviseur 
                de $a$ et $b$. \textbf{Par définition} : 
                \[ n = pgcd(a,b) \]
            \item $n$ n'est pas l'unique diviseur de $a$ et $b$ et il existe un pgcd(a,b), tel que 
                \[ n \neq pgcd(a,b) \]
        \end{itemize}
        Dans le premier cas, on sait que $n$ divise le $pgcd(a,b)$, par le \textcolor{brown}{\textbf{Lemme 1}}. 
        Dans le deuxième cas, on sait que $n$ divise $pgcd(a,b)$ puisque n'importe quel 
        nombre $n \in \mathbb{Z^*}$ peut se diviser lui-même. \vspace{1em} \\      
        Par conséquent, nous concluons que si $n$ divise $a$ et $n$ divise $b$, alors $n$ divise 
        nécessairement le plus grand commun diviseur de $a$ et $b$. \qed
    \end{Preuve*}
        

    \begin{prop}{($P^{\prime\prime}_r(a,b, n)$)}{}
        \[ n|pgcd(a,b) \implies (n|a) \land (n|b) \]
    \end{prop}

    \begin{Preuve*}{}{}
       Nous voulons montrer que si $n$ divise le plus grand commun diviseur de $a$ et $b$ \textbf{alors}, $n$ 
       divise $a$ \textbf{et} $n$ divise $b$. Nous allons montrer 
       $P^{\prime\prime}_r(a,b, n)$ par \textcolor{red}{\textit{preuve directe}}. \vspace{1em} \\
        Supposons que $n$ divise le plus grand commun diviseur de $a$ et $b$. \textbf{Alors}, $n$ 
        est un facteur de $pgcd(a,b)$ et il existe un entier $k \in \mathbb{Z}$ tel que 
        $nk = pgcd(a,b)$.  \vspace{1em} \\ 
        \textbf{Or}, s'il existe bien un nombre qui se trouve à être le plus grand commun diviseur de $a$ 
        et $b$, $n$ est alors un facteur de $a$ tout en étant un facteur de $b$. Autrement dit, 
        il est possible d'obtenir $a$ en multipliant $pgcd(a,b)$ par un entier $l \in \mathbb{Z}$  
        et il est possible d'obtenir $b$ en multiple $pgcd(a,b)$ par un eniter $m \in \mathbb{Z}$. 
        \vspace{1em} \\ 
        Similairement, il est possible d'obtenir $a$ en multipliant $n$ par $kl$ et 
        il est possible d'obtenir $b$ en multiplant $n$ par $km$ : 
        \begin{align*}
                a &= pgcd(a, b) \cdot l = nkl \\
                b &= pgcd(a, b) \cdot m = nkm 
        \end{align*}
        \textbf{Par définition}, $n$ est donc un facteur de $a$ tout en étant un facteur de $b$. 
        Ainsi, $n$ divise $a$ et $n$ divise $b$. Nous concluons que si $n$ divise $pgcd(a,b)$ 
        $n$ divise également $a$ et $b$. 
    \end{Preuve*}

    Nous venons de montrer que la proposition $P^{\prime\prime}_q(a, b, n)$ et sa 
    récriproche $P^{\prime\prime}_q(a, b, n)$ sont toutes deux vraies. Par la définition d'une 
    \textit{proposition biconditionnelle}, nous concluons que la proposition :
    \[ n|pgcd(a,b) \Leftrightarrow (n|a) \land (n|b) \]
    est vraie. Autrement dit, $P^{\prime\prime}(a, b, n) \Coloneqq$ 
    \textit{n divise le plus grand commun diviseur de $a$ et $b$ si et seulement si $n$ divise $a$ 
    et $n$ divise $b$} est vraie. 

    \begin{enumerate}
        \item[5.] Montrez que $pgcd(an; bn) = n \times pgcd(a; b)$. Reformulez la proposition en langage
        logique, puis écrivez sa preuve en explicitant chaque technique utilisée.
        (Indice : Montrez que $n \times pgcd(a; b)$ divise $pgcd(an; bn)$. Qu’en déduisez-vous ?)
    \end{enumerate}

    Soit la proposition $P(a,b,n,d, d^{\prime})$ : $pgcd(an, bn) = n \times pgcd(a,b)$, nous 
    pouvons réécrire $P(a,b,n,d, d^{\prime})$ en language logique de la façon suivante :
    \begin{align*}
        &P(a,b,n,d^{\prime})  \Coloneqq \left(d^{\prime} = pgcd(an,bn)\right) 
        \implies \left( d = n \times pgcd(a,b) \right), \\ 
        &a,b,n, d^{\prime} \in \mathbb{Z}
    \end{align*} 
    Nous allons procéder en montrant que le nombre $d^{\prime}$ divise $n \times pgcd(a,b)$ et que 
    $n \times pgcd(a,b)$ divise le nombre $d^{\prime}$, ce qui montre que les deux expressions sont 
    égales. Nous commençons par prouver le Lemme suivant. 

    \begin{Lemme}{}{}
        Si $a|b$ et $b|a$, \textbf{alors}, $a = b$ ou $a = -b$, pour tout $a, b \in \mathbb{Z}$   
    \end{Lemme}                 
    \begin{Preuve*}{}{}
        Nous procédons par \textcolor{red}{\textit{preuve directe}}. Supposons que $a|b$ et $b|a$. 
        Alors, il existe des entiers $k$ et $l \in \mathbb{Z}$ tels que 
        $ak = b$ et $bl = a$. Donc nous avons : 
        \begin{align*}
            a &= bl \\
                  &= (ak) \cdot l\\
                  &= akl\\
        \end{align*}
        Et donc, nous avons également :
        \begin{align*}
            a - akl = 0 \\ 
            a(1 -kl) = 0 \\ 
            1 - kl = 0 \\ 
            1 = kl
        \end{align*}
        Sachant que $k$ et $l$ appartiennent à $\mathbb{Z}$, les seuls nombres qui satisfont la dernière égalité 
        est $k = l = 1$ ou $k = l = -1$. 
        \begin{itemize}
            \item Si  $k = l = 1$, $a = bl = b \cdot 1 = b$. Et $b = ak = a \cdot 1 = a$
            \item Si  $k = l = -1$, $a = bl = b \cdot -1 = -b$. Et $b = ak = a \cdot -1 = -a$
        \end{itemize}
        Donc, nous concluons que si $a|b$ et $b|a$, il s'ensuit que $a = b$ ou $a = -b$. \qed
    \end{Preuve*}
        Le corollaire de ce lemme est 
        que si nous considérons uniquement des entiers $a, b, k, l \in \mathbb{N}$, 
        $a|b$ et $b|a$ implique que $a = b$. Par ailleurs, nous pouvons faire 
        ce saut logique, puisque le problème implique la notion de pgdc qui, par définition, 
        est un entier positif. 
    \begin{Lemme}{}{}
        Si $a|b$ et $b|a$, \textbf{alors}, $a = b$, pour tous $a, b \in \mathbb{N}$   
    \end{Lemme}
    Avant de montrer $P(a,b,n, d^{\prime})$, nous introduisons un autre Lemme qui nous permettra de 
    résoudre le problème. 
    \begin{Lemme}{}{}
        Si $a|c$ et $b|c$ et $pgcd(a,b)$, \textbf{alors}, $ab|c$    
    \end{Lemme}
    \begin{Preuve*}{}{}
        Supposons que $a|c$ et $b|c$ et $pgcd(a,b) = 1$. Alors, il existe des entiers $k$ et $l$ formant 
        une combinaison linéaire de $a$ et $b$ égale à $1$ ; $ak + bl = 1$ 
        (Conséquence du \textbf{Théorème de Bézout}). \textbf{Par conséquent} $cak + cbl = c$ :
        \begin{align*}
                    ak + bl &=  1 \\ 
                    c(ak + bl) &= 1 \cdot c \\ 
                    cak + cbl &= c 
        \end{align*}
        Par ailleurs, puisque $a|c$ et $b|c$, doit exister des entiers $m$ et $p$ $\in \mathbb{Z}$ 
        tels que \textcolor{myg}{$c = ma$} et \textcolor{myp}{$c = pb$}.  
        Nous avons donc $\textcolor{myp}{(pb)}ak + \textcolor{myg}{(ma)}bl = c$ : 
        \begin{align}
                \nonumber (pb)ak + (ma)bl = c \\  
                \nonumber pbak + mabl = c \\  
                \nonumber ab(pb) + ab(ml) = c \\  
                ab(pb + ml) = c  
        \end{align}
        Puisque $ab$ divise le côté gauche de l'équation (1.6) ($ab|ab(pb +ml)$), $ab$ divise nécessairement 
        le côté droit de l'équation, c'est-à-dire $c$. 
        Nous venons de montrer que si $a|c$ et $b|c$ et $pgcd(a,b) = 1$, \textbf{alors}
        $ab|c$. \qed
    \end{Preuve*}
    Nous avons prouvé le \textcolor{brown}{\textbf{Lemme 4}} en supposant que $pgcd(a,b) = 1$. Cependant, même 
    si le $pgcd$ de $a$ et $b$ n'est pas 1, nous pouvons toujours trouver un entier $n$ tel que 
    $abn = c$, car $c$ est une multiple de $a$ et $b$. Donc si $a|c$ et $b|c$, même si 
    $pgcd(a,b) \neq 1$, $ab$ divisera $c$.  
    \begin{Lemme}{}{}
        Si $a|c$ et $b|c$, \textbf{alors}, $ab|c$    
    \end{Lemme}




    \begin{prop}{($P_q(a,b, n, d^{\prime})$)}{}
    \begin{align*}
             d^{\prime} = pgcd(an, bn) \implies d^{\prime} \text{ \textbf{divise} } n \times pgcd(a,b),
             \\ a, b, n, d^{\prime} \in \mathbb{N} 
    \end{align*}       
    \end{prop}
    
    \begin{Preuve*}{}{}
        Nous procédons par \textcolor{red}{\textit{preuve directe}}. Supposons que $d^{\prime}$ est le 
        plus grand commun diviseur de $an$ et $bn$, pour $an$ et $bn \in \mathbb{N}$ ; $d^{\prime} = pgcd(an, bn)$ 
        . Et soit $d = pgcd(a,b)$ Alors, nous savons que \textcolor{myb}{$d^{\prime}$ divise toutes les 
        \textbf{combinaisons linéaires} de $an$ et $bn$}, \textbf{par la définition d'un $pgcd$}.
        Si $d$ est effectivement le $pgcd(a,b)$, alors $d$ est une combinaison linéaire de $a$ et $b$, 
        par le théorème de Bézout. Autrement dit, $d = ax + by = pgcd(a,b)$. En multipliant 
        cette combinaison linéaire ($d$) par $n$, on obtient l'équation $nd = n \times pgcd(a,b) = n(ax + by)$
        Et on peut l'expandre en \textcolor{myb}{$nd = nax + nbx$}. Cette dernière équation est simplement 
        une combinaison linéaire de $an$ et $bn$. 
        \begin{align*}
            d = pgcd(a,b) &= ax + by &\text{(Bézout)} \\ 
            n \times d &=  n(ax + by) \\ 
            n \times d &= nax + nby \\
            n \times d &= (an)x + (bn)y 
        \end{align*}
        Puisque $d^{\prime}$ divise toutes les combinaisons linéaire de $an$ et $bn$ et que 
        $nd$ est peut être reformulé en combinaison linéaire de $an$ et $bn$, nous concluons que 
        $d^{\prime}$ divise $nd$. Autrement dit, $pgcd(an ,bn)$ divise $n \times pgcd(a,b)$ 
    \end{Preuve*}

    \begin{prop}{($P_r(a,b, n, d^{\prime})$)}{}
    \begin{align*}
             d = n \times pgcd(a, b) \implies \text{ \textbf{divise} } pgcd(an, bn),
             \\ a, b, n, d \in \mathbb{N} 
    \end{align*}       
    \end{prop}


    \begin{Preuve*}{}{}
        Nous procédons par \textcolor{red}{\textit{preuve directe}}. Supposons que  $d$ est le 
        plus grand commun diviseur de $a$ et $b$, pour $a$ et $b$ $\in \mathbb{N}$. Et soit 
        $d^{\prime} = pgcd(an, bn)$. Alors, $d^{\prime}$ est une combinaison linéaire de 
        $an$ et $bn$. 
        et nous pouvons exprimer $d^{\prime}$ comme suit  $ d = anx + bny$ ou 
        \textcolor{myp}{$n(ax + by$)}.  Aisi, nous savons que $n$ divise $d^{\prime} = n(ax + by)$ \vspace{1em}. \\ 

        Prouvons maintenant que $d$ divise $d^{\prime}$. Par définition, $d$ divise 
        toutes les combinaisons linéaire de $a$ et $b$. Donc, $d$  divise un combinaison linéaire telle que 
        $ax + by$. En multipliant cette combinaison linéaire par $n$, on obtient $a(nx) + b(nx)$, ce qui est 
        aussi une combinaison linéaire de $a$ et $b$. Or, nous avons dit que $d^{\prime}$ peut être 
        reformulé en \textcolor{myp}{$n(ax + by)$} $ = a(nx) + b(nx)$, qui est toujours 
        une combinaison linéaire de $a$ et $b$. Ainsi, nous concluons que $d$ divise $d^{\prime}$. \vspace{1em} \\ 
        Nous avons montré que $n$ divise $d^{\prime}$ et que $d$ divise $d^{\prime}$. 
        Par le \textbf{\textcolor{brown}{Lemme 5}}, le produit $nd$ divise donc $d^{\prime}$. \qed   
    \end{Preuve*}


    En montrant, $P_q(a,b, n, d^{\prime})$ et $P_r(a,b, n, d^{\prime})$, nous avons montré que 
    $pgcd(an, bn)$ divise $n\times pgcd(a,b)$ et que $n \times pgcd(a, b)$ divise $pgcd(an, bn)$. 
    Par le \textbf{\textcolor{brown}{Lemme 3}}, nous concluons donc que $P(a,b,n, d^{\prime})$ tient. 
    Autrement dit : 
    \[ pgcd(an, bn) =  n \times pgcd(a, b) \]
    parce que 
    \begin{itemize}
        \item $pgcd(an, bn) \textbf{ divise }  n\times pgcd(a,n)$ \textbf{et} \\ 
        \item $n\times pgcd(a,n) \textbf{ divise } pgcd(an, bn)$
    \end{itemize}
    \qed

    \section*{Problème 1 $\quad$ $\cdot$  $\quad$ Sierpaskal}
    

    \begin{enumerate}
        \item Rappelez la définition récursive de $i \choose j$.  
    \end{enumerate}

    \begin{figure}[H]
                \[
                \begin{array}{ccccccccccccc}
                 & & & & & 1 & & & & & \\
                 & & & & 1 & & 1 & & & & \\
                 & & & 1 & & 2 & & 1 & & & \\
                 & & 1 & & 3 & & 3 & & 1 & & \\
                 & 1 & & 4 & & 6 & & 4 & & 1 & \\
                1 & & 5 & & 10 & & 10 & & 5 & & 1 \\
                \end{array}
                \]
    \caption{Triangle de Pascal }
    \end{figure}

    \begin{figure}[H]
\[
        \begin{array}{ccccccccccccccccc}
         & & & & & & & \binom{0}{0} & & & & & & & & \\
         & & & & & & \binom{1}{0} & & \binom{1}{1} & & & & & & & \\
         & & & & & \binom{2}{0} & & \binom{2}{1} & & \binom{2}{2} & & & & & & \\
         & & & & \binom{3}{0} & & \binom{3}{1} & & \binom{3}{2} & & \binom{3}{3} & & & & & \\
         & & & \binom{4}{0} & & \binom{4}{1} & & \binom{4}{2} & & \binom{4}{3} & & \binom{4}{4} & & & & \\
         & & \binom{5}{0} & & \binom{5}{1} & & \binom{5}{2} & & \binom{5}{3} & & \binom{5}{4} & & \binom{5}{5} & & & \\
        \end{array}
\]
    \caption{Représentation du Triangle de Pascal}
    \end{figure}

    \begin{Concept*}{}{}
     En observant le Triangle de Pascal, on observe deux \textbf{cas extrêmes}. Le premier cas 
    est lorsque $n \choose k$ est tel que $n \choose 0$; Il s'agit de chacune des premières entrées 
    du triangle à la rangée $n$ (considérant qu'il existe une rangée 0). Le Le second cas 
    est lorsque $n \choose k$ est tel que $n \choose n$ et donc $k = n$. Il s'agit de chacune des 
    dernières entrées du triangle à la rangé $n$.        
    \end{Concept*}


    \begin{note}{}{}
        \textbf{Par définition}, 
        $n \choose 0$ est \textbf{le nombre} de sous-ensemble de longueur $0$ il est possible
        de former en sélectionnant 
        $0$ élément d'un ensemble de longueur $n$. Dans ces conditions,
        \textbf{on peut seulement former l'ensemble vide},
        $\emptyset$, et ce \textbf{nombre} est donc $1$. 
        Par ailleurs, $n \choose n$ est le \textbf{nombre} de sous-ensemble de longueur $n$ il est possible d'obtenir 
        en sélectionnant $n$ éléments d'un ensemble de $n$ éléments. \textbf{Le seul sous-ensemble 
        possible selon ses conditions} est l'ensemble original de $n$ élément, et le nombre 
        $n \choose n$ est donc égale à $1$. 
    \end{note}

    Nous postulons alors que les \textbf{cas extrêmes} du triangle de Pascal sont de bon candidat pour 
    des \textbf{cas de base} d'une définition récursive. Considérons alors la définition partielle suivante. 
    \begin{Definitionx}{Cas de base de $C(n, k)$}{}
       $n \choose k$ $\Coloneqq$ $n \choose 0$ $= 1$ \textbf{et} $n \choose n$ $= 1$      
    \end{Definitionx}

    D'après le triangle de Pascal, nous savons que chaque entré à la rangée $n$ est égal à 
    l'entrée à la somme de la $k-1$\textit{-ième} entrée à la rangé $n-1$ et de la \textit{$k$-ième} entrée 
    à la rangée $n-1$.
    Autrement, dit 
    
    \begin{Definitionx}{Cas constructeur $C(n, k)$}{}
        \begin{center}
        $n \choose k$ =  $n-1 \choose k-1$ + $n -1 \choose k$, $k \neq n$, $k \neq 0$ 
        \end{center}        
    \end{Definitionx}

\begin{Definition}{}{}
    Le nombre \( C(n, k) \) est défini comme suit :
    \begin{align*}
            C(n, k) \Coloneqq
            \begin{cases}
                1 & \text{si } k = 0 \textbf{ ou } \\ 
                  & \text{si } k = n, \\
                C(n-1, k-1) + C(n-1, k) & \text{si } 0 < k < n.
            \end{cases}             
    \end{align*}
    
\end{Definition}

\end{multicols*}  

\end{document} 
